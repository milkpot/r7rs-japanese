% First page

\thispagestyle{empty}


\topnewpage[{
\begin{center}   {\huge\bf
アルゴリズム言語Scheme報告書改{\Huge$^{\mathbf{7}}$}}

\vskip 1ex
$$
\begin{tabular}{l@{\extracolsep{.5in}}l@{\extracolsep{.5in}}l}
\multicolumn{3}{c}{A\authorsc{LEX} S\authorsc{HINN},
J\authorsc{OHN} C\authorsc{OWAN},
A\authorsc{RTHUR} A. G\authorsc{LECKLER} (\textit{編})} \\
\\
S\authorsc{TEVEN} G\authorsc{ANZ} &
A\authorsc{LEXEY} R\authorsc{ADUL} &
O\authorsc{LIN} S\authorsc{HIVERS} \\

A\authorsc{ARON} W. H\authorsc{SU} &
J\authorsc{EFFREY} T. R\authorsc{EAD} &
A\authorsc{LARIC} S\authorsc{NELL}-P\authorsc{YM} \\

B\authorsc{RADLEY} L\authorsc{UCIER} &
D\authorsc{AVID} R\authorsc{USH} &
G\authorsc{ERALD} J. S\authorsc{USSMAN} \\

E\authorsc{MMANUEL} M\authorsc{EDERNACH} &
B\authorsc{ENJAMIN} L. R\authorsc{USSEL} &
\\
\\
\multicolumn{3}{c}{R\authorsc{ICHARD} K\authorsc{ELSEY},
W\authorsc{ILLIAM} C\authorsc{LINGER},
J\authorsc{ONATHAN} R\authorsc{EES}} \\
\multicolumn{3}{c}{\textit{(編, Revised$^{\mathit{5}}$ Report on the Algorithmic Language Scheme)}} \\
\\
\multicolumn{3}{c}{M\authorsc{ICHAEL} S\authorsc{PERBER},
R. K\authorsc{ENT} D\authorsc{YBVIG}, M\authorsc{ATTHEW} F\authorsc{LATT},
A\authorsc{NTON} \authorsc{VAN} S\authorsc{TRAATEN}} \\
\multicolumn{3}{c}{\textit{(編, Revised$^{\mathit{6}}$ Report on the Algorithmic Language Scheme)}} \\
\end{tabular}
$$
\vskip 2ex
{\it John McCarthy および Daniel Weinreb の霊前に捧ぐ}
\vskip 2.6ex
{\large \bf \today}             % *** DRAFT ***
\end{center}
}]

\clearpage

\chapter*{要約}

この報告書はプログラミング言語Schemeの定義的記述を記載しています。
Schemeは静的なスコープを持つ真正末尾再帰なLispプログラミング言語~\cite{McCarthy}の方言で、
Guy Lewis Steele~Jr. および Gerald Jay~Sussman によって発明されました。
非常に明確でシンプルな意味論を持ち、わずかな構文だけで式を構成できるよう設計されています。
手続き型、関数型、オブジェクト指向スタイルなど、
幅広い様々なプログラミングパラダイムがSchemeで簡単に表現できます。

\vest 導入部ではこの言語および報告書の歴史を簡単に述べます。

\vest 最初の3つの章ではSchemeの基礎となる考え方を示し、
その言語を説明するため、およびその言語でプログラムを書くために使う記法について述べます。

\vest \ref{expressionchapter}~章および\ref{programchapter}~章では
式、定義、プログラム、ライブラリの構文および意味論について述べます。

\vest \ref{builtinchapter}~章ではSchemeの組み込み手続き、
すなわちSchemeのデータ操作および入出力プリミティブのすべてについて述べます。

\vest \ref{formalchapter}~章では拡張BNF記法で記述されたSchemeの正式な構文を、
その表示的意味論と共に掲載します。
その後にSchemeの使用例も掲載します。

\vest 付録~\ref{stdlibraries}には
標準ライブラリとそこからエクスポートされている識別子の一覧を掲載します。

\vest 付録~\ref{stdfeatures}には
標準化されているけれどもオプショナルな処理系の機能名の一覧を掲載します。


\vest 最後に参考文献の一覧とアルファベット順の索引を掲載し、この報告書を締めくくります。

\begin{note}
\rfivers{}および\rsixrs{}から相当の部分がこの報告書に直接複写されており、
そのため\rfivers{}および\rsixrs{}の編集者をこの報告書の著者の一覧に掲載しています。
\end{note}

\todo{expand the summary so that it fills up the column.}

\vfill
\eject

\chapter*{Contents}
\addvspace{3.5pt}                  % don't shrink this gap
\renewcommand{\tocshrink}{-3.5pt}  % value determined experimentally
{\footnotesize
\tableofcontents
}

\vfill
\eject
